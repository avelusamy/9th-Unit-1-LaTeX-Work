\documentclass{amsart}
\usepackage[utf8]{inputenc}
\usepackage{amsmath,amssymb,latexsym, amsfonts,fullpage}

\title{Latex Homework 9th Grade\\ Unit 1 - Methods of Proof - Formal Style of a Proof\\ Week 2 - Structure and style of proof}
\author{Dr. Chapman and Dr. Rupel}
\date{Edited \today}

\begin{document}

\maketitle

\section{}
    Explain what is wrong with the following proof:\\ Theorem: 2 = 1\\Proof: Let \(a = b\). Then \(a^2 = ab\) so \(a^2-b^2 = ab - b^2\) which we can factor as \((a-b)(a+b) = (a-b)b\). Canceling gives \(a+b = b\) and since \(a = b\) we get \(b+b = b\). Dividing both sides by \(b\) gives \(2 = 1\).

\section{}
    Prove that for any natural numbers \(a,b\), there exists an \(n\) with \(an+b\) composite.
	\begin{proof}
	Theorem: $\forall a,b\in\mathbb{N},$ there exists an $n$ with $an+b$ composite.
	Proof: 
	\end{proof}
    
\section{}
    For each of the following, give an example and a counterexample:
    \begin{itemize}
        \item \(n!-1\) is prime for \(n \geq 3\)
        \item Any 3 distinct lines separate the plane into seven regions.  What additional assumptions are needed in order for this to be a true statement?
        \item If a rational function is bounded, then it is constant.
    \end{itemize}
    
\end{document}
