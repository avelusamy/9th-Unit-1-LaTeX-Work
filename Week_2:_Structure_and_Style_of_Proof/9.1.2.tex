\documentclass{amsart}
\usepackage[utf8]{inputenc}
\usepackage{amsmath,amssymb,latexsym, amsfonts,fullpage}
\usepackage{tikz}

\title{Latex Homework 9th Grade\\ Unit 1 - Methods of Proof - Formal Style of a Proof\\ Week 2 - Structure and style of proof}
\author{Dr. Chapman and Dr. Rupel}
\date{Edited \today}

\newcommand{\axes}
{
    \draw[<->] (-1,0) -- (5,0);
    \foreach \x in {0,1,...,4}
    \draw[-] (\x,-0.1) -- (\x,0.1);
    \foreach \y in {0,1,...,4}
    \draw[-] (-0.1,\y) -- (0.1,\y);
    \draw[<->] (0,-1) -- (0,4);   
}

\begin{document}

\maketitle

\section{}
    Explain what is wrong with the following proof:\\ Theorem: 2 = 1\\Proof: Let \(a = b\). Then \(a^2 = ab\) so \(a^2-b^2 = ab - b^2\) which we can factor as \((a-b)(a+b) = (a-b)b\). Canceling gives \(a+b = b\) and since \(a = b\) we get \(b+b = b\). Dividing both sides by \(b\) gives \(2 = 1\).

	If \( a = b\), then \((a-b)\) is 0. And we know that $\frac{0}{0}$ is indeterminate. Therefore, you can't divide it by 0.
\section{}
    Prove that for any natural numbers \(a,b\), there exists an \(n\) with \(an+b\) composite.
	\begin{proof}
	Theorem: $\forall a,b\in\mathbb{N},$ there exists an $n\in\mathbb{N}$ with $an+b$ composite.
	Proof: Without loss of generality, assume that $a$ is odd and $b$ is even. Every even number has a factor of 2. Assume \(n\) is even, then $a*n$ will be even. An even number plus an
	even number is also an even number. Assume now, without loss of generality, that $b$ is odd. If $n$ is odd, the $a*n$ will be odd. An odd number plus an odd number is an even number. Now assume that without loss of generality, that $a$ is even and $b$ is odd. If $n$ is odd, then $a*n$ will also be odd. An odd plus an odd number is an even number. Finally, assume that without loss of generality, that $a$ and $b$ are both even. Then, if $n$ is even, then $a*n$ will also be even. An even number plus an even number will also be an even number. Therefore, for any natural numbers $a,b$, there exists an $n$ with $an+b$ composite. 
	\end{proof}
    
\section{}
    For each of the following, give an example and a counterexample:
    \begin{itemize}
        \item \(n!-1\) is prime for \(n \geq 3\)
        \item Any 3 distinct lines separate the plane into seven regions.  What additional assumptions are needed in order for this to be a true statement?
        \item If a rational function is bounded, then it is constant.
    \end{itemize}

\vspace{2\baselineskip}

Claim 1:
$n!-1$ is prime for $n>=3$
\begin{itemize}
    \item Example: $n=5$, $n!-1=119$
    \item Counterexample: $n=8$, $n!-1=40319=23*1753$
\end{itemize}

\vspace{2\baselineskip}

Claim 2:
Any 3 distinct lines separate the plane into seven regions. What additional assumptions are needed in order for this to be a true statement?

\setlength\parindent{24pt} Answer: None of the lines can be parallel.

\vspace{10\baselineskip}

\setlength\parindent{48pt} Example:

    \begin{tikzpicture}
        \draw[<->] (-6,-3) -- (6,0);
        \draw[<->] (-3,-6) -- (0,6);
        \draw[<->] (-4,5) -- (5,-4);
    \end{tikzpicture}

\vspace{25\baselineskip}

\setlength\parindent{48pt} Counterexample:    

    \begin{tikzpicture}
        \draw[<->] (-5,0) -- (5,0);
        \draw[<->] (-2,5) -- (-2,-5);
        \draw[<->] (2,5) -- (2,-5);
    \end{tikzpicture}

\vspace{5\baselineskip}

Claim 3:    If a rational function is bounded, then it is constant.

\vspace{0.5\baselineskip}

\begin{tikzpicture}
    \axes
    \draw[-] (-1,3) -- (5,3);
    \draw[-] (-1,-0.5) -- (5,-0.5);
    \draw[variable=\x, domain=-1:5, smooth, blue] plot ({\x}, {2});
\end{tikzpicture}
\setlength\parindent{24pt} Example: $y=2$ 

\vspace{0.5\baselineskip}

\begin{tikzpicture}
    \axes
    \draw[-] (-1,3) -- (5,3);
    \draw[-] (-1,-0.5) -- (5,-0.5);
    \draw[variable=\x, domain=-1:5, smooth, blue] plot ({\x}, {((\x*\x)+(3*\x)+1)/((2*\x)+8)});
\end{tikzpicture}
\setlength\parindent{24pt} Counterexample: $y= (x^2+3x+1)/(2x+8)$

\end{document}