\documentclass{article}
\usepackage[utf8]{inputenc}
\usepackage{amsmath, amsfonts,fullpage}

\title{Latex Homework 9th Grade\\ Unit 1 - Methods of Proof - Formal Style of a Proof\\ Week 3 - Logic}
\author{Dr. Chapman and Dr. Rupel}
\date{Edited \today}

\begin{document}

\maketitle

\section{}
    Form a logical expression using the logical statements \(A,B,C\) and the operators \(\wedge,\vee,\neg\) , which is true if at least two of the statements are true, but false otherwise. Note: not every operator must be used, but you can use no others.

\section{}
    Simplify \(\neg(A\wedge\neg(B\wedge\neg(A\wedge\neg B)))\). In particular, your answer should only have negations applied to logical variables and not compound statements.
    
\section{}
    Prove that \(A\Rightarrow(B\wedge C)\) is equivalent to \((A\Rightarrow B)\wedge(A\Rightarrow C)\).
    
\end{document}
