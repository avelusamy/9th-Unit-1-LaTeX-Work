                                                                  \documentclass{article}
\usepackage[utf8]{inputenc}
\usepackage{amsmath, amsfonts,fullpage,tikz}
\usepackage{amsthm}

\title{Latex Homework 9th Grade\\ Unit 1 - Methods of Proof - Formal Style of a Proof\\ Week 4 - Set Theory}
\author{Dr. Chapman and Dr. Rupel}
\date{Edited \today}
\newcommand{\sm}{\setminus}
\begin{document}

\maketitle

\section{}
    How many subsets of the set \(\{0,1,2,3,4,5,6\}\) have elements which sum to 15?
    You should list them all.

\section{}
    Let $A$ and $B$ be subsets of a set $Z$.
    Prove that $(Z\sm A)\cup(Z\sm B)=Z\sm(A\cap B)$.
    \begin{proof}
     Let $x \in (Z\sm A)\cup(Z\sm B)$. If $x\in Z\sm A$, then $x\in A$.
    \end{proof}
\section{}
    Consider sets $A$, $B$, and $C$.
    Prove that $A\times(B\cup C)=(A\times B)\cup(A\times C)$.
	\begin{proof}
	Let $x\in A$, $y\in(B\cup C)$, $b\in B$, $c\in C$. $A\times(B\cup C)$ returns a set of ordered pairs ($x$,$y$). $A\times B$ gives ordered pairs ($x$,$b$), $A\times C$ gives ordered pairs ($x$, $c$). $B\cup C$ is the set of elements in B or C. This implies $b\in B \implies b\in(B\cup C)$ and $c\in C\implies c\in(B\cup C)$, meaning that $B\subseteq (B\cup C)$ and $C\subseteq (B\cup C)$. Because of this, $y$ can be either $b$ or $c$. Therefore, $(A\times B)\cup(A\times C)\subseteq A\times(B\cup C)$. It has already been shown that $y$ can be $b$ or $c$. $(A\times B)\cup(A\times C)$ will give ($x$,$b$) or ($x$,$c$). Therefore, $A\times(B\cup C)\subseteq(A\times B)\cup(A\times C)$. Since $A\times(B\cup C)$ and $(A\times B)\cup(A\times C)$ are subsets of each other, they are equivalent. Hence, $A\times(B\cup C)=(A\times B)\cup(A\times C)$. 
	\end{proof}
\end{document}
